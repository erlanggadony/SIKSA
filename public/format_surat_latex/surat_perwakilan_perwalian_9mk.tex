\documentclass[12pt]{letter}
\usepackage[legalpaper, top=25mm,bottom=15mm,left=35mm,right=18mm]{geometry}
\usepackage{mailmerge}

\begin{document}

\mailfields{semester,thnAKademik,namaPemohon,prodiPemohon,npmPemohon,namaWakil,prodiWakil,npmWakil,alasan,kodeMK1,namaMK1,sks1,kodeMK2,namaMK2,sks2,kodeMK3,namaMK3,sks3,kodeMK4,namaMK4,sks4,kodeMK5,namaMK5,sks5,kodeMK6,namaMK6,sks6,kodeMK7,namaMK7,sks7,kodeMK8,namaMK8,sks8,kodeMK9,namaMK9,sks9}

\mailrepeat{

	\begin{center}
	{\Large \textbf{FORMULIR\\
	PERWALIAN YANG DIWAKILKAN}}
	\end{center}
	\vspace{0.25cm}
	Semester \hspace{1.5cm}: \field{semester} \\
	Tahun Akademik : \field{thnAkademik}\\

	{\footnotesize \textbf{IDENTITAS MAHASISWA YANG PERWALIANNYA DIWAKILKAN}}\\
	Nama \hspace{2.4cm}: \field{namaPemohon} \\
	Program Studi \hspace{0.8cm}: \field{prodiPemohon}\\
	NPM \hspace{2.5cm}: \field{npmPemohon}	\\
	{\footnotesize \textbf{IDENTITAS MAHASISWA YANG DIBERI KUASA PERWALIAN}}\\
	Nama \hspace{2.4cm}: \field{namaWakil} \\
	Program Studi \hspace{0.8cm}: \field{prodiWakil}\\
	NPM \hspace{2.5cm}: \field{npmWakil}\\

	Alasan tidak bisa hadir pada saat perwalian :\\
	\field{alasan}\\
	Mata kuliah yang diambil saat FRS : \\
	\begin{tabular}{|l|l|l|l|}
		\hline
		\textbf{No.}&\textbf{Kode MK}&\textbf{Nama Mata Kuliah}&\textbf{Sks}\\ \hline
		1&\field{kodeMK1}&\field{namaMK1}&\field{sks1}\\ \hline
		2&\field{kodeMK2}&\field{namaMK2}&\field{sks2}\\ \hline
		3&\field{kodeMK3}&\field{namaMK3}&\field{sks3}\\ \hline
		4&\field{kodeMK4}&\field{namaMK4}&\field{sks4}\\ \hline
		5&\field{kodeMK5}&\field{namaMK5}&\field{sks5}\\ \hline
		6&\field{kodeMK6}&\field{namaMK6}&\field{sks6}\\ \hline
		7&\field{kodeMK7}&\field{namaMK7}&\field{sks7}\\ \hline
		8&\field{kodeMK8}&\field{namaMK8}&\field{sks8}\\ \hline
		9&\field{kodeMK9}&\field{namaMK9}&\field{sks9}\\ \hline
	\end{tabular}

\textbf{\underline{LAMPIRAN}} :
\begin{enumerate}
	\item Fotokopi KTM mahasiswa yag menerima kuasa
\end{enumerate}

		\begin{flushright}
			Bandung, \field{tanggal}
		\end{flushright}
Tanda Tangan \hspace{7.3cm}Tanda Tangan \\
Mahasiswa yang memberi kuasa \hspace{4cm} Mahasiswa yang menerima kuasa \\
\begin{flushright}
		\end{flushright}
			\vspace{1cm}

	(...........................................)\hspace{4.6cm} (...........................................)\\

	\textbf{Mengetahui dosen wali, \hspace{4.8cm} Menyetujui Wakil Dekan I}
	\vspace{2cm}

	(...........................................)\hspace{4.8cm}(\field{wdI})
	\thispagestyle{empty}
		\newpage

}


\end{document}
