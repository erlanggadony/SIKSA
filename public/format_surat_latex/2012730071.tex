\documentclass[11pt,a4paper]{report}
\usepackage{graphicx}
\begin{document}
	\title {Penyebab Ledakan Penduduk dan Cara Untuk Menanggulanginya}
	\author {Dony Erlangga - 2012730071}
	\date{} \maketitle
	Ledakan penduduk adalah pertumbuhan penduduk yang tidak terkendali di suatu negara dan terjadi secara bertahap dan cepat. Pada gambar di bawah ini terlihat pertumbuhan penduduk yang mulai menaik secara perlahan yang dimulai sejak tahun 1750 hingga tahun 1900. Pada tahun 1910 terjadi pertumbuhan dua kali lebih banyak dari pada tahun sebelumnya yang berjumlah 10 juta jiwa menjadi 20 juta jiwa. Kemudian pada tahun 1940 tejadi peningkatan yang lebih tajam lagi dari 20 juta jiwa bertambah menjadi 50 juta jiwa. Angka-angka ini terus bertambah seiring dengan bergantinya tahun. Hingga puncaknya terjadi pada tahun 1970 dimana jumlah penduduk saat itu mencapai lebih dari 80 juta jiwa.\\\\\	
	\begin{figure}[htbp]
		\centering
			\includegraphics[width=9cm,height = 7cm]{./HumanPop.jpg}
			\caption{Tabel Populasi Manusia}
			\label{Gambar 1 : Tabel Populasi Manusia}
	\end{figure}
	\\
	Ledakan penduduk dapat disebabkan oleh beberapa faktor. Faktor-faktor yang sangat berpengaruh diantaranya : 
	\begin{itemize}
		\item Penyebaran penduduk yang tidak merata,
		\item Tingginya pernikahan usia dini,
		\item Angka kelahiran yang tinggi,
		\item Program KB yang tidak terlaksana dengan baik.
	\end{itemize}
	
	Berdasarkan penyebab terjadinya ledakan penduduk, pada tahun 1970 terjadi apa yang disebut sebagai \textit{baby booming} yaitu dimana angka kelahiran pada saat itu sangat tinggi. Hal ini juga dipengaruhi oleh prinsip \textit{banyak anak banyak rezeki}, sehingga setiap keluarga seperti "berlomba-lomba" untuk memiliki anak sebanyak mungkin dengan harapan setelah anak-anak mereka dewasa orang tua akan mendapat "rezeki" dari sang anak. Memang benar adanya seperti itu. Tetapi jika tidak dibarengi dengan pendidikan yang layak bagi sang anak, rasanya mendapatkan "rezeki" dari sang anak setelah dia dewasa hanya sekedar menjadi angan-angan belaka.\\\\\
	\\
	Untuk menyelesaikan masalah ledakan penduduk dibutuhkan upaya penanggulangan yang tepat dari pemerintah. Upaya-upaya yang dapat dilakukan diantaranya : 
	\begin{itemize}
		\item Pembatasan kelahiran bayi dengan program keluarga berencana melalui semboyan \textit{vatus warga},
		\item Menggalakkan program KB,
		\item Pembatasan usia perkawinan,
		\item Pembatasan tunjangan anak bagi PNS,
		\item Pelaksanaan program tansmigrasi.
	\end{itemize}
\end{document}

