\documentclass[12pt]{letter}
\usepackage[legalpaper, top=25mm,bottom=15mm,left=35mm,right=18mm]{geometry}
\usepackage{mailmerge}

\begin{document}

\mailfields{semester,thnAkademik,namaPemohon,prodiPemohon,npmPemohon,namaWakil,prodiWakil,npmWakil,dosenWali,alasan,kodeMK,namaMK,sks,tanggal}

\mailrepeat{

	\begin{center}
	{\Large \textbf{FORMULIR\\
	PERWALIAN YANG DIWAKILKAN}}
	\end{center}
	\vspace{0.25cm}
	Semester \hspace{1.5cm}: \field{semester} \\
	Tahun Akademik : \field{thnAkademik}\\

	{\footnotesize \textbf{IDENTITAS MAHASISWA YANG PERWALIANNYA DIWAKILKAN}}\\
	Nama \hspace{2.4cm}: \field{namaPemohon} \\
	Program Studi \hspace{0.8cm}: \field{prodiPemohon}\\
	NPM \hspace{2.5cm}: \field{npmPemohon}	\\
	{\footnotesize \textbf{IDENTITAS MAHASISWA YANG DIBERI KUASA PERWALIAN}}\\
	Nama \hspace{2.4cm}: \field{namaWakil} \\
	Program Studi \hspace{0.8cm}: \field{prodiWakil}\\
	NPM \hspace{2.5cm}: \field{npmWakil}\\

	Alasan tidak bisa hadir pada saat perwalian :\\
	\field{alasan}\\
	Mata kuliah yang diambil saat FRS : \\
	\begin{tabular}{|c|l|l|l|}
		\hline
		\textbf{No.}&\textbf{Kode MK}&\textbf{Nama Mata Kuliah}&\textbf{Sks}\\ \hline
		1&\field{kodeMK}&\field{namaMK}&\field{sks}\\ \hline
	\end{tabular}

\textbf{\underline{LAMPIRAN}} :
\begin{enumerate}
	\item Fotokopi KTM mahasiswa yag menerima kuasa
\end{enumerate}

		\begin{flushright}
			Bandung, \field{tanggal}
		\end{flushright}
Tanda Tangan \hspace{7.3cm}Tanda Tangan \\
Mahasiswa yang memberi kuasa \hspace{4cm} Mahasiswa yang menerima kuasa \\
\begin{flushright}
		\end{flushright}
			\vspace{1cm}

	(\field{namaPemohon})\hspace{4.6cm} (\field{namaWakil})\\

	\textbf{Mengetahui dosen wali, \hspace{4.8cm} Menyetujui Wakil Dekan I}
	\vspace{2cm}

	(\field{dosenWali})\hspace{4.8cm}(Dr.rer.nat. Cecilia Esti Nugraheni)
	\thispagestyle{empty}
		\newpage

}
 
\mailentry{Genap,2016/2017,MARCELL TRIXIE ALEXANDER,Teknik Informatika,2014730003,NAMAWAKIL,PRODIWAKIL,NPMWAKIL,Mariskha Tri Adithia,ALASAN,1A,A,1,19 May 2017}
\end{document}
